%% Generated by Sphinx.
\def\sphinxdocclass{report}
\documentclass[letterpaper,10pt,english]{sphinxmanual}
\ifdefined\pdfpxdimen
   \let\sphinxpxdimen\pdfpxdimen\else\newdimen\sphinxpxdimen
\fi \sphinxpxdimen=.75bp\relax
\ifdefined\pdfimageresolution
    \pdfimageresolution= \numexpr \dimexpr1in\relax/\sphinxpxdimen\relax
\fi
%% let collapsible pdf bookmarks panel have high depth per default
\PassOptionsToPackage{bookmarksdepth=5}{hyperref}

\PassOptionsToPackage{booktabs}{sphinx}
\PassOptionsToPackage{colorrows}{sphinx}

\PassOptionsToPackage{warn}{textcomp}
\usepackage[utf8]{inputenc}
\ifdefined\DeclareUnicodeCharacter
% support both utf8 and utf8x syntaxes
  \ifdefined\DeclareUnicodeCharacterAsOptional
    \def\sphinxDUC#1{\DeclareUnicodeCharacter{"#1}}
  \else
    \let\sphinxDUC\DeclareUnicodeCharacter
  \fi
  \sphinxDUC{00A0}{\nobreakspace}
  \sphinxDUC{2500}{\sphinxunichar{2500}}
  \sphinxDUC{2502}{\sphinxunichar{2502}}
  \sphinxDUC{2514}{\sphinxunichar{2514}}
  \sphinxDUC{251C}{\sphinxunichar{251C}}
  \sphinxDUC{2572}{\textbackslash}
\fi
\usepackage{cmap}
\usepackage[T1]{fontenc}
\usepackage{amsmath,amssymb,amstext}
\usepackage{babel}



\usepackage{tgtermes}
\usepackage{tgheros}
\renewcommand{\ttdefault}{txtt}



\usepackage[Bjarne]{fncychap}
\usepackage[,numfigreset=1,mathnumfig]{sphinx}

\fvset{fontsize=auto}
\usepackage{geometry}


% Include hyperref last.
\usepackage{hyperref}
% Fix anchor placement for figures with captions.
\usepackage{hypcap}% it must be loaded after hyperref.
% Set up styles of URL: it should be placed after hyperref.
\urlstyle{same}

\addto\captionsenglish{\renewcommand{\contentsname}{Contents:}}

\usepackage{sphinxmessages}
\setcounter{tocdepth}{1}



\title{Loadupch}
\date{Jun 18, 2024}
\release{0.0.1}
\author{Unit Electronics}
\newcommand{\sphinxlogo}{\sphinxincludegraphics{Logo-UNIT_Web-04-800x182.png}\par}
\renewcommand{\releasename}{Release}
\makeindex
\begin{document}

\ifdefined\shorthandoff
  \ifnum\catcode`\=\string=\active\shorthandoff{=}\fi
  \ifnum\catcode`\"=\active\shorthandoff{"}\fi
\fi

\pagestyle{empty}
\sphinxmaketitle
\pagestyle{plain}
\sphinxtableofcontents
\pagestyle{normal}
\phantomsection\label{\detokenize{index::doc}}


\begin{sphinxadmonition}{warning}{Warning:}
\sphinxAtStartPar
Loadupch is freely available for use. However, the software is currently under development and may contain bugs.
\end{sphinxadmonition}

\sphinxAtStartPar
Loadupch is a software tool specifically designed for uploading firmware to the CH552 microcontroller.
Developed in Python, this application leverages the PyUSB library to facilitate communication with the device.

\sphinxAtStartPar
Loadupch is based on the loader implementation by Stefan Wagner, available at \sphinxhref{https://github.com/wagiminator}{Stefan Wagner’s GitHub}.
Originally inspired by the \sphinxcode{\sphinxupquote{chprog.py}} project found at \sphinxhref{https://github.com/wagiminator/CH552-USB-OLED/tree/main/software/cdc\_i2c\_bridge/tools}{chprog.py on GitHub},
Loadupch enhances the original by introducing a graphical user interface, making it significantly more user\sphinxhyphen{}friendly.

\begin{figure}[htbp]
\centering
\capstart

\noindent\sphinxincludegraphics{{loadupch}.png}
\caption{Loadupch interface}\label{\detokenize{index:id1}}\end{figure}

\sphinxAtStartPar
This documentation is divided into the following sections:

\sphinxstepscope


\chapter{Installation}
\label{\detokenize{installation:installation}}\label{\detokenize{installation::doc}}
\begin{sphinxadmonition}{warning}{Warning:}
\sphinxAtStartPar
Loadupch is freely available for use. However, the software is currently under development and may contain bugs.
\end{sphinxadmonition}


\section{Requirements}
\label{\detokenize{installation:requirements}}
\sphinxAtStartPar
Loadupch requires  \sphinxhref{https://www.python.org/}{Python 3.6 or higher} and the following libraries:

\begin{sphinxVerbatim}[commandchars=\\\{\}]
pip\PYG{+w}{ }install\PYG{+w}{ }pyusb
\end{sphinxVerbatim}

\begin{sphinxVerbatim}[commandchars=\\\{\}]
pip\PYG{+w}{ }install\PYG{+w}{ }tkinter
\end{sphinxVerbatim}


\section{Loadupch Installation}
\label{\detokenize{installation:loadupch-installation}}
\sphinxAtStartPar
For easy installation, you can use the following command:

\begin{sphinxVerbatim}[commandchars=\\\{\}]
pip\PYG{+w}{ }install\PYG{+w}{ }loadupch
\end{sphinxVerbatim}


\section{Driver Installation}
\label{\detokenize{installation:driver-installation}}
\begin{sphinxuseclass}{sphinx-tabs}
\sphinxAtStartPar
Windows

\sphinxAtStartPar
To install the driver, you can use the Zadig software. Download the latest version of \sphinxcode{\sphinxupquote{Zadig}}. You can download it from the official website.
\begin{quote}


\end{quote}

\sphinxAtStartPar
Ubuntu

\sphinxAtStartPar
If you use Linux, you may need to install the following packages:

\begin{sphinxVerbatim}[commandchars=\\\{\}]
sudo\PYG{+w}{ }apt\PYGZhy{}get\PYG{+w}{ }install\PYG{+w}{ }libusb\PYGZhy{}1.0\PYGZhy{}0\PYGZhy{}dev\PYG{+w}{ }libudev\PYGZhy{}dev
\end{sphinxVerbatim}

\end{sphinxuseclass}

\section{Uninstallation}
\label{\detokenize{installation:uninstallation}}
\sphinxAtStartPar
In the case that software is not useful for you, you can uninstall it using the following command:

\begin{sphinxVerbatim}[commandchars=\\\{\}]
pip\PYG{+w}{ }uninstall\PYG{+w}{ }loadupch
\end{sphinxVerbatim}

\sphinxstepscope


\chapter{Usage}
\label{\detokenize{usage:usage}}\label{\detokenize{usage::doc}}
\sphinxAtStartPar
To use the Loadupch, you can run the following command:

\begin{sphinxVerbatim}[commandchars=\\\{\}]
python\PYG{+w}{ }\PYGZhy{}m\PYG{+w}{ }loadupch
\end{sphinxVerbatim}

\sphinxAtStartPar
The software will open a window with the interface. You can select the device press the “Connect” button.

\begin{figure}[htbp]
\centering
\capstart

\noindent\sphinxincludegraphics[width=0.500\linewidth]{{loadupch_connect}.png}
\caption{Loadupch interface}\label{\detokenize{usage:id1}}\label{\detokenize{usage:figure-connect}}\end{figure}

\begin{sphinxadmonition}{note}{Note:}
\sphinxAtStartPar
Before connect the device, press the “boot” button on the CH552 board.
\end{sphinxadmonition}

\begin{figure}[htbp]
\centering
\capstart

\noindent\sphinxincludegraphics[width=0.800\linewidth]{{CH552_boot}.png}
\caption{Cocket Nova CH552 button boot}\label{\detokenize{usage:id2}}\label{\detokenize{usage:figure-flash}}\end{figure}

\sphinxAtStartPar
The interface will show the message “Device connected” and you can select the firmware file and press the “flash firmware” button.

\begin{figure}[htbp]
\centering
\capstart

\noindent\sphinxincludegraphics[width=0.500\linewidth]{{device_found}.png}
\caption{Device found message}\label{\detokenize{usage:id3}}\label{\detokenize{usage:figure-message}}\end{figure}

\sphinxstepscope


\chapter{Upload firmware}
\label{\detokenize{flash:upload-firmware}}\label{\detokenize{flash::doc}}
\sphinxAtStartPar
For the upload firmware, press the “flash firmware” button.

\begin{sphinxadmonition}{note}{Note:}
\sphinxAtStartPar
Only \sphinxcode{\sphinxupquote{.bin}} files are supported.
\end{sphinxadmonition}

\sphinxAtStartPar
Automatically the software open the file dialog, you can select the firmware file and press the “open” button.

\begin{figure}[htbp]
\centering
\capstart

\noindent\sphinxincludegraphics[width=0.500\linewidth]{{loadupch_firmware}.png}
\caption{Flash firmware dialog}\label{\detokenize{flash:id1}}\label{\detokenize{flash:figure-flash-firmware}}\end{figure}

\sphinxAtStartPar
The software will show the message “Firmware flashed” and the device will be ready to use.

\begin{figure}[htbp]
\centering
\capstart

\noindent\sphinxincludegraphics[width=0.500\linewidth]{{success_flas}.png}
\caption{Firmware flashed message}\label{\detokenize{flash:id2}}\label{\detokenize{flash:figure-success}}\end{figure}


\section{Verify firmware loaded}
\label{\detokenize{flash:verify-firmware-loaded}}
\sphinxAtStartPar
To verify the firmware loaded, press the “verify firmware” button.

\begin{figure}[htbp]
\centering
\capstart

\noindent\sphinxincludegraphics[width=0.500\linewidth]{{verify_firmware}.png}
\caption{Verify firmware dialog}\label{\detokenize{flash:id3}}\label{\detokenize{flash:figure-verify}}\end{figure}

\sphinxAtStartPar
The software will show the message “Firmware verified” if the firmware is loaded correctly.

\sphinxstepscope


\chapter{Examples}
\label{\detokenize{examples:examples}}\label{\detokenize{examples::doc}}
\begin{sphinxadmonition}{note}{Note:}
\sphinxAtStartPar
There exists a repository with examples for the Cocket Nova CH552 board, which you can find here:
\begin{quote}


\end{quote}
\end{sphinxadmonition}

\sphinxAtStartPar
The repository provides examples for developing software in C using the SDCC compiler for the CH552 microcontroller. It serves as an excellent resource for both beginners and experienced developers, offering versatile and affordable solutions for code development.

\sphinxAtStartPar
Clone the repository and follow the instructions to compile and upload the firmware to the CH552 microcontroller:
\begin{quote}


\end{quote}


\section{Upload Firmware Using Loadupch}
\label{\detokenize{examples:upload-firmware-using-loadupch}}
\sphinxAtStartPar
To upload the firmware using Loadupch, follow the steps below:
\begin{enumerate}
\sphinxsetlistlabels{\arabic}{enumi}{enumii}{}{.}%
\item {} 
\sphinxAtStartPar
Clone the repository with the examples:

\end{enumerate}

\begin{sphinxVerbatim}[commandchars=\\\{\}]
git\PYG{+w}{ }clone\PYG{+w}{ }https://github.com/UNIT\PYGZhy{}Electronics/CH55x\PYGZus{}SDCC\PYGZus{}Examples.git
\end{sphinxVerbatim}
\begin{enumerate}
\sphinxsetlistlabels{\arabic}{enumi}{enumii}{}{.}%
\setcounter{enumi}{1}
\item {} 
\sphinxAtStartPar
Open the Loadupch software:

\end{enumerate}

\begin{sphinxVerbatim}[commandchars=\\\{\}]
python\PYG{+w}{ }\PYGZhy{}m\PYG{+w}{ }loadupch
\end{sphinxVerbatim}
\begin{enumerate}
\sphinxsetlistlabels{\arabic}{enumi}{enumii}{}{.}%
\setcounter{enumi}{2}
\item {} 
\sphinxAtStartPar
Press the \sphinxcode{\sphinxupquote{\textasciigrave{}boot\textasciigrave{}}} button on the CH552 board and connect the device to the USB port of your computer.

\end{enumerate}

\begin{figure}[htbp]
\centering
\capstart

\noindent\sphinxincludegraphics[width=0.800\linewidth]{{CH552_boot}.png}
\caption{Cocket Nova CH552 button boot}\label{\detokenize{examples:id1}}\label{\detokenize{examples:figure-flash2}}\end{figure}
\begin{enumerate}
\sphinxsetlistlabels{\arabic}{enumi}{enumii}{}{.}%
\setcounter{enumi}{3}
\item {} 
\sphinxAtStartPar
Press the \sphinxcode{\sphinxupquote{\textasciigrave{}Connect\textasciigrave{}}} button on the Loadupch interface.

\end{enumerate}

\begin{figure}[htbp]
\centering
\capstart

\noindent\sphinxincludegraphics[width=0.500\linewidth]{{loadupch_connect}.png}
\caption{Loadupch interface}\label{\detokenize{examples:id2}}\label{\detokenize{examples:figure-connect2}}\end{figure}
\begin{enumerate}
\sphinxsetlistlabels{\arabic}{enumi}{enumii}{}{.}%
\setcounter{enumi}{4}
\item {} 
\sphinxAtStartPar
Select the firmware file from the \sphinxcode{\sphinxupquote{Software/examples/Blink}} repository and press the \sphinxcode{\sphinxupquote{\textasciigrave{}flash firmware\textasciigrave{}\textasciigrave{}}} button.

\end{enumerate}

\begin{figure}[htbp]
\centering
\capstart

\noindent\sphinxincludegraphics[width=0.600\linewidth]{{example_blink}.png}
\caption{Path to example blink}\label{\detokenize{examples:id3}}\label{\detokenize{examples:figure-example}}\end{figure}
\begin{enumerate}
\sphinxsetlistlabels{\arabic}{enumi}{enumii}{}{.}%
\setcounter{enumi}{5}
\item {} 
\sphinxAtStartPar
The software will show the message \sphinxcode{\sphinxupquote{\textasciigrave{}Firmware flashed\textasciigrave{}}} and the device will be ready to use.

\item {} 
\sphinxAtStartPar
Finally, press the \sphinxcode{\sphinxupquote{\textasciigrave{}run programmer Exit\textasciigrave{}}} button on the Loadupch interface.

\end{enumerate}

\begin{figure}[htbp]
\centering
\capstart

\noindent\sphinxincludegraphics[width=0.800\linewidth]{{led}.png}
\caption{Blink example}\label{\detokenize{examples:id4}}\label{\detokenize{examples:figure-flash-firmware}}\end{figure}

\sphinxstepscope


\chapter{Unpaired Device}
\label{\detokenize{error:unpaired-device}}\label{\detokenize{error::doc}}
\sphinxAtStartPar
There are several reasons why the device might not be connected:
\begin{itemize}
\item {} 
\sphinxAtStartPar
The device is not connected to the USB port.

\item {} 
\sphinxAtStartPar
The device is not in boot mode.

\item {} 
\sphinxAtStartPar
The system does not have the necessary permissions to access the device.

\item {} 
\sphinxAtStartPar
The driver is not installed.

\end{itemize}

\sphinxAtStartPar
To solve the problem, you can try the following steps:
\begin{enumerate}
\sphinxsetlistlabels{\arabic}{enumi}{enumii}{}{.}%
\item {} 
\sphinxAtStartPar
Check if the device is connected to the USB port.

\item {} 
\sphinxAtStartPar
Check if the device is in boot mode.

\item {} 
\sphinxAtStartPar
Check if the system has the necessary permissions to access the device.

\end{enumerate}

\sphinxAtStartPar
If the problem persists, you can try the following steps:
\begin{enumerate}
\sphinxsetlistlabels{\arabic}{enumi}{enumii}{}{.}%
\item {} 
\sphinxAtStartPar
Check if the driver is installed.

\end{enumerate}

\sphinxAtStartPar
Use Zadig to install the driver:

\begin{figure}[htbp]
\centering
\capstart

\noindent\sphinxincludegraphics[width=0.900\linewidth]{{driver}.png}
\caption{Zadig interface}\label{\detokenize{error:id1}}\label{\detokenize{error:figure-zadig}}\end{figure}
\begin{enumerate}
\sphinxsetlistlabels{\arabic}{enumi}{enumii}{}{.}%
\item {} 
\sphinxAtStartPar
Go to the “Options” menu and select “List All Devices”.

\item {} 
\sphinxAtStartPar
Locate the device with USB MODULE VID = 0x4348 and PID = 0x55E0.

\end{enumerate}

\begin{figure}[htbp]
\centering
\capstart

\noindent\sphinxincludegraphics[width=0.900\linewidth]{{zadig_device}.png}
\caption{Device selected}\label{\detokenize{error:id2}}\label{\detokenize{error:figure-device}}\end{figure}
\begin{enumerate}
\sphinxsetlistlabels{\arabic}{enumi}{enumii}{}{.}%
\setcounter{enumi}{2}
\item {} 
\sphinxAtStartPar
Select the driver “libusb\sphinxhyphen{}win32 (v1.2.6.0)” and press the “Install Driver” button.

\end{enumerate}

\begin{figure}[htbp]
\centering
\capstart

\noindent\sphinxincludegraphics[width=0.900\linewidth]{{zadig_driver}.png}
\caption{Driver selected}\label{\detokenize{error:id3}}\label{\detokenize{error:figure-install-driver}}\end{figure}

\begin{figure}[htbp]
\centering
\capstart

\noindent\sphinxincludegraphics[width=0.900\linewidth]{{zadig_success}.png}
\caption{Device installed}\label{\detokenize{error:id4}}\label{\detokenize{error:figure-device-success}}\end{figure}

\sphinxAtStartPar
After that, the device will be recognized by the system.


\chapter{Indices and tables}
\label{\detokenize{index:indices-and-tables}}\begin{itemize}
\item {} 
\sphinxAtStartPar
\DUrole{xref,std,std-ref}{genindex}

\item {} 
\sphinxAtStartPar
\DUrole{xref,std,std-ref}{modindex}

\item {} 
\sphinxAtStartPar
\DUrole{xref,std,std-ref}{search}

\end{itemize}



\renewcommand{\indexname}{Index}
\printindex
\end{document}